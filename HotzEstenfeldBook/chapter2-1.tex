\section{The finite automaton, regular sets in $X^*$,\ $REG(X^*)$}

Let $G = (V, E)$ be a finite, oriented graph, $X$ a finite set and $\alpha =
(\pathcat{G}, X^*,\alpha_1, \alpha_2)$ a functor with $\alpha_1 : V \to \{ X^*
\},\ \alpha_2 : \pathcat{G} \to X^*$.

(Remark by the translator: Here the free monoid $X^*$ is regarded as a category
$X^* = (\{X^*\}, X^*, Q, Z, \cdot \})$ where $\cdot$ is the monoid operation
(word concatenation). Words are treated as morphisms with source and target
$X^*$. Klingt komisch, is aber so.)

\begin{definition}[nondeterministic finite automaton]
$\automat{A} = (G, X^*, \alpha)$ is called a {\bf nondeterministic finite
automaton}.
\end{definition}

If $S, F \in V$ are points of the graph $G$, we call $\automat{A} = (G, X^*,
S, F, \alpha)$ a finite automaton with start and final states or shortly a {\bf
finite acceptor}.

In the following we will use the terms acceptor and automaton as synonyms.

If the finite automaton works over the free monoid $X^*$ we write shorter just
$X$ instead of $X^*$, otherwise we specify the monoid explicitly.

\begin{definition}[accepted set]
If we define 
\[ \pathcat{G}(S, F) := \{ w \in \pathcat{G}\ |\ Q(w) \in S \wedge Z(w)
\in F \}, \]
then \[ L_{\automat{A}} := \alpha_2(\pathcat{G}(S, F)) \] 
is called the set {\bf accepted by} the automaton. 
\end{definition}

We will also write shortly $\alpha$ instead of $\alpha_2$ and $\pathcat{}(S,F)$
instead of $\pathcat{G}(S,F)$.

\begin{definition}[regular language over free monoid]
Let $X$ be an alphabet. \[ REG(X^*) := \{ L \subset X^*\ | \mbox{ there
exists a finite automaton } \automat{A} \mbox{ with } L = \automatlang{A} \}
\]

$REG(X^*)$ is the set of {\bf regular languages} over the free monoid $X^*$.
\end{definition}

Remark: We defined here the finite automaton via its ''state graph''. Most
often, the definition is given using the ''next state relation'' as follows:

\[ \delta = \{ (a, P_1, P_2) \in X \times V \times V)\ | \]
\[ \mbox{there exists an edge } e \mbox{ with } Q(e) = P_1, Z(e) = P_2 \mbox{
and } \alpha(e) = a \in X \}. \]

$\delta$ my be regarded as a relation between $X \times V$ and $V$ where $X$ is
the input alphabet and $V$ the state set of the automaton \[ \automat{B} = (X,
V, \delta, S, F) \]

The elements of $V$ denote the current state of the automaton $\automat{B}$.

If the automaton $\automat{B}$ is in state $z \in V$ and reads the symbol $x
\in X$ then it changes into state $z' \in V$ where $(x, z, z') \in \delta$. If
there doesn't exist such a $z'$ the automaton halts.

This interpretation can be visualized as follows:

FIGURE

Let's return to out definition of the finite automaton. We explain its
working based on our definition:

The automaton $\automat{A} = (G, X, S, F, \alpha)$ may be interpreted as a
nondeterministic algorithm. The points of graph $G$ define the possible states
of the algorithm, the elements of $X$ are the input alphabet.

The nondeterministic automaton $\automat{A}$ which reads a symbol $x \in X$
while in state $P \in V$ changes into state $P'$ if there exists an edge $e$
from $P$ to $P'$ with label $\alpha(e) = x$. If the graph has no such edge
originating in $P$ the automaton is set ''out of service''.

A finite acceptor accepts a word $w \in X^*$ if there exists a path from a point
in $S$ to a point in $F$ which is labeld with $w$.

Let's consider some examples for finite automata:

Example 1: Let $X = \{ a, b \}$ and $L = \{ (a b)^{2n}\ |\ n \in \mathbb{N} \}$.
It holds: $L \in REG(\{a, b\}^*)$.

The following acceptor accepts $L$ (exercise):

$\automat{A} = (G, \{a, b\}, {1}, {1}, \alpha)$.

FIGURE

Example 2: Lexical analysis, check for special characters.

In every programming language there exist special character combinations
(reserved words) that mark certain program actions. These have to be identified
during the lexical analysis. We give a finite acceptor which realizes such a check for a selection
of reserved words:

Let the set of reserved words be 
$\{$ 'BEGIN', 'END', 'ELSE', 'IF', FI', 'FOR', 'INTEGER', 'THEN', 'LOOP',
'POOL', 'PROCEDURE' $\}$.

The following acceptor accepts this set:

FIGURE

The images of the edges under the mapping $\alpha$ are shown as edge labels. The
points of the graph are the ovals with their labels. Start and final states are
given by $S = \{$ START $\}$ and $F = \{$ STOP $\}$.

The labels of the points are chosen such that one can see the information stored
by the automaton.

Now we want to prove some properties of $REG(X^*)$. To do that, we need
some basic properties for finite automata.

\begin{lemma}
Let $\automat{A} = (G, X, S, F, \alpha)$ be a finite automaton. Then there
exists a finite automaton $\automat{A'} = (G', X, S', F', \alpha')$ such that
$card(S') = card(F') = 1$ and $\automatlang{A} = \automatlang{A'}$.
\end{lemma}

An automaton with a single start state is called {\bf initial}.

Proof: If $card(S) = card(F) = 1$ we are done.

(Comment by translator: The proof in the book is completely unreadable because
of all these tildes, primes, indices etc. Therefore it is reformulated here.) 

Let $card(S) > 1$ or $card(F) > 1$.

1. Add new edges leaving the new start state $S'$:

Define the set of all edges leaving an old start state by
\[ OUT := \{ e \in E \where Q(e) \in S \} \]
Add the following new edges to the graph:
\[ OUT' := \{ e' = (S', Z(e)) \where e \in OUT,\ \alpha'(e') := \alpha(e) \} \]

2. Add new edges reaching the new final state $F'$:

Define the set of all edges reaching an old final state by
\[ IN := \{ e \in E \where Z(e) \in F \} \]
Add the following new edges to the graph:
\[ IN' := \{ e' = (Q(e), F') \where e \in IN,\ \alpha'(e') := \alpha(e) \} \]

To each new edge we assign the same label as the edge from which it has been
derived.

The new automaton $\automat{A'} = (G', X, \{S'\}, \{F'\}, \alpha')$ is defined
by the graph $G' = (V \cup \{ S', F' \}, E \cup OUT' \cup IN')$
and the new labeling $\alpha'$ which is identical to $\alpha$ for all existing edges and is defined as shown above for the new edges.

It is easily shown that $\automatlang{A'} = \automatlang{A}$.

\begin{lemma}
Let $\automat{A} = (G, X, S, F, \alpha)$ be a finite automaton. Then there
exists an automaton $\automat{A'} = (G', X, S', F', \alpha')$ with $\alpha'(e)
\in X\ \forall e \in E(G)$ and $\automatlang{A} = \automatlang{A'}$.
\end{lemma}

Proof: We ''split'' all edges according to their labels.

\begin{enumerate}
  \item Let $e \in E$ with $\alpha_2(e) = x_1 \cdots x_k,\ k > 1, x_i \in X$.\\
  Remove edge $e$ and add new edges $e'_1,
  \ldots, e'_k$ and new points $P'_1, \ldots, P'_{k-1}$ such that $(Q(e),
  e'_1, \ldots, e'_k, Z(e)) \in \pathcat{G'}$ and define a new graph $G' =
  (V', E')$. The labeling of the new edges is defined by $\alpha'(e'_i) := x_i$
  for $i = 1, \ldots, k$.
  
  Then $\alpha_2(e) = \alpha'_2(e'_1) \cdots \alpha'_2(e'_k)$.
  
  \item Let $e \in E$ be an edge labeled with $\epsilon$.
  \begin{enumerate}
    \item  (Remove $\epsilon$-loops) \\
    If $Q(e) = Z(e) : E' := E - e$
    \item (Remove
    $\epsilon$-edges which cannot be continued to a longer path)\\
    If there is
    not $e' \in E$ with $Q(e') = Z(e') : E' := E - e$
    \item  (Skip $\epsilon$-edges that can be continued and remove the
    $\epsilon$-edge)\\
    If there exists an edge $e' \in E$ with $Q(e') = Z(e) : E'' := E - e$.
    Add new edges: $E' := E'' \cup \{
    \tilde{e} \where Q(\tilde{e}) = Q(e), Z(\tilde{e}) = Z(e'),\
    alpha'(\tilde{e}) := \alpha(e') \}$
  \end{enumerate}
  If in step (b) or (c) the target of the edge is a final state, then add the
  source of the edge to the set of final states.
  
  Continue this algorithm inductively until no more $\epsilon$-edges remain in
  the graph. The algorithm terminates because the point and edge sets are
  finite. For the new automaton $\automat{A'}$ that results from this algorithm
  holds: $\automatlang{A'} = \automatlang{A}$.
\end{enumerate}

If we apply this algorithm to our automaton from example 1, we obtain:

FIGURE

Now we want to prove some closure properties of $REG(X^*)$.

\begin{theorem}
Let $L, L' \in REG(X^*)$. Then the union $L \cup L'$ and the intersection $L
\cap L'$ both are regular.
\end{theorem}

Proof: Let $L = \automatlang{A}$ and $L' = \automatlang{B}$ with automata \[
\automat{A} = (G_A, X, S_A, F_A, \alpha) \] and \[ \automat{B} = (G_B, X, S_B,
F_B, \beta).\]
We may assume that the edge and point sets of both automata graphs are disjoint.

\begin{enumerate}
  \item Closure under union: Define
	\[ \gamma_2(e) := \left\{
		\begin{array}{l} 
		\alpha_2(e),\ e \in E(G_A) \\
		\beta_2(e),\ e \in E(G_B)
		\end{array}
	 \right. \]

	Then the automaton $\automat{C} = (G_A \cup G_B, X, S_A \cup S_B, F_A \cup F_B)$
	accepts the language $\automatlang{A} \cup \automatlang{B}$.
	
	\item Closure under intersection: Define $G' = (V', E')$ where
	\[ V' = V_A \times V_B \]
	\[ E' = \{ (e_A, e_B) \in E_A \times E_B \where \alpha_2(e_A) = \beta_2(e_B)
		 \}. \]
	By lemma 2 we may assume that the edge labels are all single symbols from $X$.
	
	We define the new labeling $\delta_2$ by \[ \delta_2 : E' \to X,\
	\delta_2((e_A, e_B) = \alpha_2(e_A)). \]
	
	For the automaton $\automat{A'} = (G', X, S_A \times S_B, F_A \times F_B,
	\delta)$ then holds: $\automatlang{G'} = \automatlang{A} \cap \automatlang{B}$
	and this automaton is called the {\bf cartesian product} of $\automat{A}$ and
	$\automat{B}$.
\end{enumerate}










































