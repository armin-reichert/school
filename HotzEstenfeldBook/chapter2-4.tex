\section{Right-linear languages, $r{-}LIN(X^*)$}

In this section we investigate an additional method for defining special subsets
of $X^*$. We use the mechanism introduced in chapter 1.6, the Chomsky grammars.

We consider only a very restricted type of productions and obtain the {\bf
right-linear} grammars.

\begin{definition}[right-linear Chomsky grammar]
A Chomsky grammar $G = (N, X, P, S)$ is called
\begin{eqnarray*}
\mbox{\bf right-linear} & & \mbox{if } P \subset N \times (X \cdot N \cup X) \\
\mbox{\bf left-linear}  & & \mbox{if } P \subset N \times (N \cdot X \cup X)
\end{eqnarray*}
\end{definition}

$X$ is the {\bf terminal alphabet}.

As in chapter 1.6 we have the notions of {\em derivation} and {\em generated
language}.

\begin{definition}[right-linear languages]
\[ r{-}LIN(X^*) = \{ L \subset X^* \mid \mbox{there exists a right-linear
grammar $G$ with } L = L(G) \} \]
is the class of {\bf right-linear languages} in $X^*$.
\end{definition}

In the same way one defines the class of {\bf left-linear languages}.

The following theorem holds:

\begin{theorem}
\[ REG(X^*) = r{-}LIN(X^*) \]
\end{theorem}

Proof:

''$\supset$'': Let $L \in r{-}LIN(X^*)$ with $L = L(G),\ G = (N, X, P, S)$.

We construct a graph $G' = (V, E)$ with vertices $V = N \cup \{F\}, F \notin N$
and we use the productions of the grammar as edge set $E$:
\begin{eqnarray*}
p: v \to x v' \in P & \Rightarrow & e:  v \edge{x} v' \in E \\
p: v \to x \in P & \Rightarrow & e : v \edge{x} F \in E
\end{eqnarray*} 

(Translator remark: $e: v \edge{x} v'$ is just a shorter notation for $Q(e) =
v, \ Z(e) = v',\ \alpha(e) = x$)

This defines a finite acceptor $\fa{A} = (G', X, S, F, \alpha)$.

It is easily seen that
\[ w \in \pathcat{G'} \Rightarrow Q(w) \derives{G} \alpha(w) \cdot Z(w) \mbox{
 , if } Z(w) \in N\]

For paths $w \in \pathcat{G'}(S, F)$ we get $\alpha(w) \in L$.

From this it follows $\falang{A} \subset L(G)$.

Let $w \in L(G)$ be a word generated by the right-linear grammar $G$, then there
exists a sequence of derivation steps
\[ S \dderives{G} x_1 v_1 \dderives{G} x_1 x_2 v_2 \dderives{G} \ldots
\dderives{G} x_1 \cdots x_{n-1} v_{n-1} \dderives{G} w \]

Then there exists a path 
\[ w' = \big(S, (S,x_1 v_1), (v_1, x_2 v_2), \ldots, (v_{n-1}, x_{n-1} v_{n-1}),
(v_{n-1}, x_n), F\big) \in \pathcat{G'}(S, F) \]
and the label of this path is
\[ \alpha(w') = x_1 \cdots x_n = w \]

This means, the word $w$ is accepted by the finite acceptor, $w \in \falang{A}$.

Together, we get $\falang{A} = L(G)$.

''$\subset$'':

























