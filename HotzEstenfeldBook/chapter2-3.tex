\section{Sets recognizable by homomorphisms, $REC(X^*)$}

In this section we investigate a third possibility for characterizing subsets of
$X^*$.

\begin{definition}
\begin{eqnarray*}
 REC(X^*) & := & \{ L \subset X^* \mid \mbox{ there exists a
finite monoid } H \\
& & \mbox{ and a monoid homomorphism }\mu : X^* \to H \\
& & \mbox{ with } L = \mu^{-1}(T), T \subset H \}
\end{eqnarray*}
is the class of {\bf recognizable sets} over $X$.
\end{definition}

\begin{lemma}
\[ REG(X^*) \subset REC(^*) \]
\end{lemma}

Proof: Let $L \in REG(X^*)$, then there exists a complete, deterministic finite
automaton $\fa{A} = (G, X, S, F, \alpha)$ with graph $G = (V, E)$ accepting $L$.

We define \[ H := Map(V,V) := \{ f : V \to V \mid f \mbox{ is a mapping} \} \]

and define a homomorphism $\mu : X^* \to H$ as follows:

Let $\mu$ be the homomorphic continuation of mapping $\mu'$ where
\begin{eqnarray*}
& \mu'(x)(P) = R & \\
& \Leftrightarrow & \\
& \mbox{ there exists an edge }e \in E\mbox{ with }Q(e) = P, Z(e) = R\mbox{ and
}\alpha(e) = x \in X &
\end{eqnarray*}

We define further \[ T := \{ f \in H \mid f(S) \in F \} \]

Then it holds $\falang{A} = \mu^{-1}(T)$ (exercise).

\begin{lemma}
\[ REC(X^*) \subset REG(X^*) \]
\end{lemma}

Proof: Sei $L \in REC(X^*)$, thus let $H$ be a finite monoid and $\mu : X^* \to
H$ be a monoid homomorphism and $L = \mu^{-1}(T), T \subset H$.

We construct a graph $G = (V, E)$:

Vertex set: $V := H$.

Edges: $E := \{ (h, a) \mid h \in H, a \in X \}$ with $Q((h,a)) = h, Z((h,a))
= h \cdot \mu(a)$.



























