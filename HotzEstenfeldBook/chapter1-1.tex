\section{Notations, basic terminology}

In this first section we want to define the elementary terminology that is
used throughout the whole book. We use the usual notions
\[ \mathbb{N} = \{ 0, 1, 2, \ldots \}\ \mbox{for the natural numbers} \]
\[ \mathbb{Z} = \{ \ldots, -2, -1, 0, 1, 2, \ldots \}\ \mbox{for the integer
numbers} \]
\[ \mathbb{Q} = \{ \frac{a}{b} \mid a,b \in \mathbb{Z}, b \neq 0 \}\ \mbox{for
the rational numbers } \]

For the set operations we use $\cup$ for the union and $\cap$ for the
intersection. Also $A \subset B$, $a \in A$, $a \not\in A$, $\bar{A}$, $A - B$,
$A \times B$ and $\emptyset$ have their usual meaning. For the power set of a
set $A$ we write $2^A$ or $Pot(A)$. $Card(A)$ denotes the cardinality of $A$.
Logical implication is denoted by $\Rightarrow$.

{\bf Mappings} are denoted as $f : A \rightarrow B$, in which case $f$ is a
total mapping. We write $Q(f) = A, Z(f) = B$. 

(Translator remark: $Q$ stands for ''Quelle'' (source)
and $Z$ for ''Ziel'' (target).)

If $f: A \rightarrow B,\ g : B \rightarrow C$ are mappings, then $f \circ g : A
\rightarrow C$ is the composed mapping which one gets by applying \boldmath $f$
{\bf first and then} $g$ \unboldmath, i.e. \[(f \circ g)(a) = g(f(a))\]
If $f:A \rightarrow B$ and $C \subset A$, then $f(C) = \{ f(c) \mid c \in C \}$.

(Translator remark: Note that this is the other way round as the usual
definition of function composition but makes perfect sense here in the category theoretic
treatment.)

A subset $R \subset A \times B$ is called a {\bf relation} between $A$ and $B$.
$R_f \subset A \times B$ with $R_f = \{ (a,b) \mid f(a) = b \}$ is the relation
{\bf induced by the mapping} f or the {\bf graph} of $f$.

Let $f : A \rightarrow B$ be a mapping, $A_1 \subset A$ and $g : A_1 \rightarrow
B$ a mapping. $f$ is called the {\bf continuation} of $g$ if $f(a_1) = g(a_1),
a_1 \in A_1$. In this case we also write $f \mid _{A_1} = g$ (in words: $f$
{\bf restricted to} $A_1$).

A {\bf semi-group} consists of a set $M$ and an associative operation on that
set, usually denoted as a multiplication. If a semi-group is commutative, we
also use ''$+$'' instead of ''$\cdot$''.

A semi-group is a \index{{\bf monoid}} if $M$ contains a neutral element. We
often denote it with $1_M$ or shortly $1$. In the commutative case we often write $0$
instead of $1$.

For $A,B \subset M$ we denote by $A \cdot B = \{ a b \mid a \in A, b \in B \}$
the {\bf complex product} of $A$ and $B$.

$A \subset M$ is a {\bf submonoid} of $M$ if the following holds: $1_M \in A$
and $A$ is closed under the operation of $M$.

For a set $A$, the set $A^*$ defined as follows, is the smallest submonoid of
$M$ which contains $A$. 

More specific,
\[ A^* = \bigcap_{M' \in M(A)} M'	\]
where \[ M(A) = \{ M' \subset M \mid M' \mbox{ is a submonoid of } M, A \subset
M' \} \]
It is easy to see that
\[ A^* = \bigcup_{n \geq 0} A^n\ \mbox{with}\ A^0 = \{1\}\ \mbox{and}\ A^{n+1} =
A^n \cdot A \]
In the same sense the notion $A^+ = A^* - \{1\}$ is defined for semi-groups. $A$
is called the {\bf generation system} of $A^*$ and $A^+$ resp.

A special meaning for us is assigned to the set of ''words'' (string) over a
fixed alphabet $A$. We understand as words the finite sequences of elements from
the alphabet $A$ as for example $(a,b,d,a,c)$ for alphabet $A = \{ a,b,c,d \}$.

We define
\[
WORD(A) := \{\epsilon\} \cup A \cup (A \times A) \cup (A \times A \times A) \cup
\ldots
\]
as the {\bf set of words (strings)} over $A$. The symbol $\epsilon$ denotes the
{\bf empty word} over $A$, that is $A^0 = \{\epsilon\}$.

If $v, w \in WORD(A)$ then $v \cdot w$ is the word which you get by
concatenating $v$ and $w$, formally:
\[ v = (a_1,\ldots, a_k), w = (a_{k+1}, \ldots, a_n) \Rightarrow v \cdot
w = (a_1, \ldots, a_n) \]
With this operation, $WORD(A)$ becomes a monoid which is usually also denoted
with $A^*$. 

This is slightly inconsistent because for the first definition of
the $*$-operator it holds $(A^*)^* = A^*$ but for the second usage of the
$*$-operator it holds $(A*)^* \neq A^*$.

The following example should clarify that: 

Let $A = \{a,b,c\}$ and let $(a,b,a)$ and $(b,a) \in A^*$.
\[(a,b,a)\cdot(b,a) = (a,b,a,b,a) \in A^*,\ \mbox{but}\]
\[((a,b,a),(b,a)) \in (A^*)^*\ \mbox{but}\ \notin A^*\]

Instead of $(a)$ we write just $a$. In this sense it holds $A \subset A^*$. This
also holds in the sense of the first definition of $A^*$.

If $w = (w_1, \ldots, w_n)$ we call $|w| := n$ the {\bf length} of
$w$. Obviously it holds: \[|w \cdot v| = |w| + |v| \quad\mbox{ and }\quad
|\epsilon| = 0\]

The {\bf mirror word} $w^R$ of a word $w = (w_1,\ldots,w_n)$ is the word
$(w_n,\ldots,w_1)$. It holds: $(w \cdot v)^R = v^R \cdot w^R$ and $\epsilon^R =
\epsilon$.

In $A^*$ the {\bf reduction rules} hold, i.e.
\begin{enumerate}
  \item $w \cdot x = w \cdot y \Rightarrow x = y$
  \item $x \cdot w = y \cdot w \Rightarrow x = y$
\end{enumerate}

We define {\bf left} and {\bf right quotient} for sets of words $X, Y$:
\[ X^{-1} \cdot Y = \{ w\ |\ \exists x \in X,\ y \in Y\ \mbox{with}\ x \cdot w =
y \} \] and 
\[ X \cdot Y^{-1} = \{ w\ |\ \exists x \in X,\ y \in Y\ \mbox{with}\ w \cdot y =
x \} \]

Because of the reduction rules it holds:
\[ \{w\}^{-1} \cdot \{v\}\ \mbox{and}\ \{w\}^{-1} \cdot \{v\}\ \mbox{are
either empty or contain a single element.} \]

If $\{w\}^{-1} \cdot \{v\}$ is not empty, we call $w$ a {\bf prefix} of $v$. If
$\{w\} \cdot \{v\}^{-1} \not= \emptyset $, we call $v$ a {\bf suffix} of $w$.

In the future we will always write just $w$ instead of $\{w\}$ and also $w$
{\bf is prefix of} $v$, if $w^{-1} \cdot v \not= \emptyset$.

