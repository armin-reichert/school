\section{Homomorphy and equivalence of finite automata}

We consider again the finite automaton from chapter 2.1 and investigate the
question of structural relationship between finite automata.

We tackle the question how finite automata are related which accept the same
language. We will formalize this by the existence of certain functors between
finite automata.

For making statements about decidability wrt.\ accepted languages it is an
advantage to be able to determine a ''simplest'''' finite acceptor. We will
prove the existence of such a ''simplest'' automaton.

First, we consider some simple decidability questions answered by the following
theorems:

\begin{theorem}
Let $\fa{A}$ be a finite automaton. Then the questions $\lang{A} = \emptyset?$
and $\lang{A} = X^*?$ are decidable.
\end{theorem}

Proof:

''$\lang{A} = \emptyset$?'':

Because of $\lang{A} = \emptyset \Leftrightarrow \pathcat{G}(S, F) =
\emptyset$ this is a decidable problem (reachability in a finite graph). The
formal proof is up to the reader.

''$\lang{A} = X^*$?'':

We gave in chapter 2.1 an effective procedure for constructing a deterministic,
finite automaton $\fa{A}'$ from an automaton $\fa{A}$. From this deterministic
automaton, one can construct an automaton $\fa{B}$ accepting the complement of
$\lang{A}$. From $\lang{B} = \emptyset \Leftrightarrow \lang{A} = X^*$,
and 1) follows the claim.

\begin{definition}
$\fa{A}$ is called {\bf weakly-equivalent} to an automaton $\fa{B}$ if the
accepted languages are equal: \[ \lang{A} = \lang{B} \]
\end{definition}

\begin{theorem}[decidability of weak equivalence]
For finite automata $\fa{A}$ and $\fa{B}$ it is decidable if $\lang{A} =
\lang{B}$.
\end{theorem}

Proof: Without loss of generality we may assume that $\fa{A}$ and $\fa{B}$ are
complete and deterministic. 

From $\fa{A}$ and $\fa{B}$ using our constructions we can directly give automata
for 
\[\bar{\lang{A}}, \bar{\lang{B}}, \bar{\lang{A}} \cap
\lang{B}, \lang{A} \cap \bar{\lang{A}}\]
and also for 
\[ \lang{C} := (\bar{\lang{A}} \cap \lang{B}) \cup (\lang{A} \cap
\bar{\lang{A}}) \]

Because $\lang{A} = \lang{B} \Leftrightarrow \lang{C} = \emptyset$ which
is decidable (theorem 1) we get the proof of theorem 2.

We consider now the homomorphy between finite automata. Structure-preserving
mappings between finite automata are called {\bf automata homomorphisms}.

\begin{definition}
Let $\fa{A}, \fa{A'}$ be finite automata. A functor $\beta = (\beta', \beta'')$
is called an {\bf automata homomorphism}, if
\[ \beta' = (\pathcat{G}, \pathcat{G'}, \beta'_1, \beta'_2) \]
is a functor,
\[ \beta'': X^* \to X'^*	\]
is a monoid homomorphism, and the axioms (A1) and (A2) hold:
\begin{itemize}
  \item[(A1)] The following diagram commutes:
  
  \begin{tikzcd}[column sep=large]
  \pathcat{G} \arrow[r, "\alpha"] \arrow[d, "\beta'"] & X^* \arrow[d, "\beta''"]
  \\
  \pathcat{G'} \arrow[r, "\alpha'"] & X'^*
  \end{tikzcd}
  
  \item[(A2)] $\beta'_1(S) = S'$ and $\beta'_1(F) = F'$. 
\end{itemize}

$\beta$ is called {\bf length-preserving} $\Leftrightarrow \beta'_2(e) \in
E(G')$ for each edge $e \in E(G)$.

$\beta$ is called {\bf non-expanding} $\Leftrightarrow \beta_2'(e) \in E(G')
\cup \{ 1_e \mid e \in E(G') \}$
\end{definition}

In the following we will always identify $\beta'$ with $\beta$.

It holds:

\begin{theorem}
Given an automata homomorphism $\beta: \fa{A} \to \fa{A'}$. If
\[\beta_2(\pathcat{G}(S,F)) = \pathcat{G'}(S',F')\]
then $\beta''(\lang{A}) = \lang{A'}$.
\end{theorem}

Proof: 

''$\subset$'':

Let $u \in \lang{A}$, then there exists a path $\pi \in \pathcat{G}(S, F)$
labelled with $u$, i.e. $\alpha(\pi) = u$. Because $\beta_2(\pi) \in
\pathcat{G'}(S',F')$ by assumption, it follows $\alpha'(\beta_2(\pi)) \in
\lang{A'}$.

Because of the commutativity of the diagram it follows $ \alpha'(\beta_2(\pi))
= \beta''(\alpha(\pi))$.
\[ \Rightarrow \beta''(\lang{A}) \subset \lang{A'} \]

''$\supset$'':

Let $u' \in \lang{A'}$. Then there exists a path $\pi' \in \pathcat{G'}(S', F')$
that is labeled with $u'$: $\alpha'(\pi') = u'$.

Because $\beta_2$ is surjective on $\pathcat{G'}(S', F')$ it follows: there
exists a path $\pi$ with $\beta_2(\pi) = \pi'$ and because $\beta$ is a functor
it follows $\alpha(\pi) \in \lang{A}$.

By definition, $\beta''(\alpha(\pi)) = \alpha'(\beta_2(\pi)) = \alpha'(\pi') =
u'$, therefore $u' \in \beta''(\lang{A})$.
\[ \Rightarrow  \beta''(\lang{A}) \supset \lang{A'} \]

Together this proves \[ \lang{A'} = \beta''(\lang{A}) \]

\begin{definition}[Reduction, closed homomorphism]
Let $\fa{A}$ and $\fa{A'}$ be finite automata.

A homomorphism $\beta : \fa{A \to \fa{A'}}$ is called a {\bf reduction} from
$\fa{A}$ to $\fa{A'}$ if \[ \beta_2(\pathcat{G}(S, F)) = \pathcat{G'}(S', F') \]
and $\beta'' = id_{X^*}$.

$\beta$ is a {\bf length-preserving reduction}, if it is a reduction and the
homomorphisms is length-preserving.

$\beta$ is a {\bf non-expanding reduction}, if it is a
reduction and the homomorphisms is non-expanding.

$\beta$	 is called a {\bf closed homomorphism} or short: {\bf closed}, if it is
a homomorphism and it holds \[ \beta_1^{-1}(S') = S \quad\wedge\quad
\beta_1^{-1}(F') = F
\]

In analogy, one defines a {\bf closed reduction}.
\end{definition}

We immediately get the following corollary to theorem 3:

\begin{corollary}
\[ \beta : \fa{A} \to \fa{A'} \mbox{ reduction } \Rightarrow \lang{A} =
\lang{A'} \]
\end{corollary}

Our goal will be to transfer equivalent automata into each other using chains of
reductions.

To reach this goal, we will need a number of lemmata.

\begin{lemma}
Let $\fa{A}, \fa{B}, \fa{C}$ be complete, deterministic finite automata and let
$\beta_{\fa{A}} : \fa{C} \to \fa{A}$ and $\beta_{\fa{B}} : \fa{C} \to \fa{B}$ be
closed reductions. Then there exists a finite automaton $\fa{D}$ and reductions
$\gamma_{\fa{A}}$ and $\gamma_{\fa{B}}$ such that the following diagram
commutes:

\begin{center}
\begin{tikzcd}
&	\fa{C} \arrow[dl, "\beta_{\fa{A}}"'] \arrow[dd, dashrightarrow, "\delta"]
\arrow[dr, "\beta_{\fa{B}}"] \\
\fa{A} \arrow[dr, dashrightarrow, "\gamma_{\fa{A}}"'] 	& & \fa{B} \arrow[dl,
dashrightarrow, "\gamma_{\fa{B}}"]\\
& \fa{D} 
\end{tikzcd}
\end{center}
\end{lemma}

Proof: The idea of the proof is as follows: We create from $\fa{C}$ a new
automaton $\fa{D}$ by putting those vertices and edges into classes which will
be identified under the reductions $\beta_{\fa{A}}$ or $\beta_{\fa{B}}$. In that
way we will get the reduction $\delta$. The difficulty that arises in
identifying the edges is that the labelling of the edges has to be respected.

To realize out idea we define equivalence relations on the edges and vertices of
the graph $G_{\fa{C}} = (V_{\fa{C}}, E_{\fa{C}})$ of $\fa{C}$:

Let $v_1, v_2 \in V_{\fa{C}}$ be vertices (states). Define
\begin{eqnarray*}
v_1 \faequiv{A} v_2 & : \Longleftrightarrow & \beta_{\fa{A}}(v_1) =
\beta_{\fa{A}}(v_2) \\
v_1 \faequiv{B} v_2 & : \Longleftrightarrow & \beta_{\fa{B}}(v_1) =
\beta_{\fa{B}}(v_2)
\end{eqnarray*}

In analogy we define such an equivalence relation for the edges of the graph
$G_{\fa{C}}$.

We define now an equivalence relation on the vertices (states) of the automaton
$\fa{C}$. For $v, v' \in V_{\fa{C}}$:
\begin{eqnarray*}
v \equiv v' & :\Leftrightarrow & \mbox{there is a chain of states } v = v_1,
v_2, \ldots, v_k = v'\mbox{ with } \\
& & v_i \faequiv{A} v_{i+1}\mbox{\quad or\quad }v_i \faequiv{B} v_{i+1},\quad i
= 1, \ldots, k
\end{eqnarray*} 

Obviously this defines an equivalence relation on the vertex (state) set
$V_{\fa{C}}$. We do this in the same way for the edges.

We get equivalence classes
\begin{eqnarray*}
\left[ v \right] & := & \{ v' \in V_{\fa{C}} \mid v \equiv v' \} \\
\left[ e \right] & := & \{ e' \in E_{\fa{C}} \mid e \equiv e' \}
\end{eqnarray*}

We define the sets of equivalence classes by
\begin{eqnarray*}
\bar{V} & := & \{ [v] \mid v \in V_{\fa{C}} \} \\ 
\bar{E} & := & \{ [e] \mid e \in E_{\fa{C}} \} 
\end{eqnarray*}

Claim: With $Q([e]) := [Q(e)]$ and $Z([e]) := [Z(e)],\ e \in E_{\fa{C}}$ we get
well-defined source and target mappings for the edges of the graph of
the automaton $\fa{C}$.

Proof. Let $e' \in [e]$. To prove that $Q(e') \in [Q(e)]$, we use induction over
the length of the chain $e = e_1, \ldots, e_k = e'$ where $s_i \faequiv{A}
s_{i+1}$ or $s_i \faequiv{B} s_{i+1}$ for all $i = 1,\ldots,k-1$. 

It is sufficient to show that the claim is correct for each single step. Because
$\beta_{\fa{A}}$ and $\beta_{\fa{B}}$ are functors, it holds
\[ e_i \faequiv{A} e_{i+1} \Rightarrow Q(e_i) \faequiv{A} Q(e_{i+1}) \]
and
\[ e_i \faequiv{B} e_{i+1} \Rightarrow Q(e_i) \faequiv{B} Q(e_{i+1}) \]

From this it follows that $Q$ and $Z$ are well-defined on the equivalence
classes.

Let $\bar{G} = (\bar{V}, \bar{E})$ be the graph defined by the sets of vertex
and edge equivalence classes. Define the labelling $\alpha_{\fa{D}} : \bar{E}
\to X^*$ for the edges of this graph by
\[ \alpha_{\fa{D}}([e]) := \alpha_{\fa{C}}(e) \]

This mapping is well-defined. 

Proof: Because $\beta_{\fa{A}}$ is a reduction, it holds:
\[ \alpha_{\fa{A}}(\beta_{\fa{A}}(e) = \beta''(\alpha_{\fa{C}}(e)) =
\alpha_{\fa{C}}(e) \]

This means that for each $e' \in [e]_{\fa{A}}$ we have
$\alpha_{\fa{A}}(\beta_{\fa{A}}(e')) = \alpha_{\fa{C}}(e)$. This our claim is
correct if $e \faequiv{A} e'$. The same holds for $e \faequiv{B} e'$. Together
we get that this holds for all $e \equiv e'$ so $\alpha_{\fa{D}}$ is
well-defined.

To complete the definition of the automaton $\fa{D}$ we have to define start and
final state sets.
\[ \fa{D} := (\bar{G}, X, [S_{\fa{C}}], [F_{\fa{C}}], \alpha_{\fa{D}}) \]

with final state set $[F_{\fa{C}}] := \{ [f] \mid f \in F_{\fa{C}} \}$.

Claim: The canonical mapping $e \to [e],\ v \to [v]$ is a reduction.

Proof: Exercise (Hint: use that automaton is complete and deterministic)

Finally we define the reductions $\gamma_{\fa{A}}$ and $\gamma_{\fa{B}}$ by
\begin{eqnarray*}
\gamma_{\fa{A}}(v) & := & [\beta_{\fa{A}}^-1](v),\quad v \in V_{\fa{A}} \\
\gamma_{\fa{B}}(v) & := & [\beta_{\fa{B}}^-1](v),\quad v \in V_{\fa{B}} \\
\gamma_{\fa{A}}(e) & := & [\beta_{\fa{A}}^-1](e),\quad e \in E_{\fa{A}} \\
\gamma_{\fa{B}}(e) & := & [\beta_{\fa{B}}^-1](e),\quad e \in E_{\fa{B}} \\
\end{eqnarray*}
\qed

Remark: The requirement of our lemma, that the automata have to be complete, is
essential, as the following example shows:

FIGURE

It holds: $\lang{A} = \lang{B} = \lang{C} = \{ a^n \mid n \in \mathbb{N} \}
\cup \{ b^n \mid n \in \mathbb{N} \}$.

From our construction it follows $4 \equiv 2 \equiv 1 \equiv 3$ and the
automaton $\fa{D}$ has the form

FIGURE

But $\lang{D} = \{ a, b \}^* \neq \lang{C}$. It follows:

There exists no reduction from $\fa{C}$ to $\fa{D}$ because \fa{A}, \fa{B} and
\fa{C} are not complete {\it and} deterministic.

The motivation for lemma 1 can be easily seen in the following diagram:

Let

\begin{center}
\begin{tikzcd}
& \fa{A}_2 \arrow[dl] \arrow[dr] & & \fa{A}_4 \arrow[dl] \arrow[dr] & & \fa{A}_6
\arrow[dl] \\
\fa{A}_1 & & \fa{A}_3 && \fa{A}_5
\end{tikzcd}
\end{center}

be a chain of reductions. Then we can construct automata $\fa{A}_7, \fa{A}_8$
and reductions like

\begin{center}
\begin{tikzcd}
& \fa{A}_2 \arrow[dl] \arrow[dr] & & \fa{A}_4 \arrow[dl] \arrow[dr] & & \fa{A}_6
\arrow[dl] \\
\fa{A}_1 \arrow[dr, dashed] & & \fa{A}_3 \arrow[dl, dashed] && \fa{A}_5
\arrow[dldl, dashed] \\
& \fa{A}_7 \arrow[dr, dashed] & \\
& & \fa{A}_8
\end{tikzcd}
\end{center}

such that the diagram can be ''shortened'' to

\begin{center}
\begin{tikzcd}
& \fa{A}_1 \arrow[dr] & & \fa{A}_6 \arrow[dl] \\
& & \fa{A}_8
\end{tikzcd}
\end{center}

\begin{lemma}
Let $\gamma_{\fa{A}} : \fa{A} \to \fa{D}$ and $\gamma_{\fa{B}} : \fa{B} \to
\fa{D}$ be length-preserving reductions.

Then there exists an automaton $\fa{C}$ and reductions $\beta_{\fa{A}}$ and
$\beta_{\fa{B}}$ such that the following diagram commutes:
\begin{center}
\begin{tikzcd}
& \fa{C} \arrow[dl,dashed,"\beta_{\fa{A}}"'] \arrow[dd,dashed]
\arrow[dr,dashed,"\beta_{\fa{B}}"]
\\
\fa{A} \arrow[dr,"\gamma_{\fa{A}}"'] & & \fa{B} \arrow[dl,"\gamma_{\fa{B}}"] \\
& \fa{D}
\end{tikzcd}
\end{center}
\end{lemma}

\begin{proof}
We construct the automaton 
\[ \fa{C} = (G_{\fa{C}}, X, S_{\fa{C}}, F_{\fa{C}}, \alpha_{\fa{C}}) \]
with vertex (state) set
\[ V_{\fa{C}} = V_{\fa{A}} \times V_{\fa{B}} \]
and edge set
\[ E_{\fa{C}} = \{ (e, e') \in E_{\fa{A}} \times E_{\fa{B}} \mid
\gamma_{\fa{A}}(e) = \gamma_{\fa{B}}(e') \} \]
start states
\[ S_{\fa{C}} = S_{\fa{A}} \times S_{\fa{B}} \]
final states
\[ F_{\fa{C}} = F_{\fa{A}} \times F_{\fa{B}} \]
and labelling
\[ \alpha_{\fa{C}}((e, e')) = \alpha_{\fa{A}}(e) \] 

$\beta_{\fa{A}}$ and $\beta_{\fa{B}}$ are the projections onto $\fa{A}$ and
$\fa{B}$.

Claim: $\beta_{\fa{A}}$ and $\beta_{\fa{B}}$ are reductions.

Without loss of generality we show the claim for $\beta_{\fa{A}}$ only.

\begin{enumerate}
\item $\alpha_{\fa{A}}(\beta_{\fa{A}}((e,e'))) = \alpha_{\fa{A}}(e) =
\beta''_{\fa{A}}(\alpha_{\fa{C}})(e, e') = \beta''_{\fa{A}}(\alpha_{\fa{A}})(e)$

With $\beta''_{\fa{A}}$ the equation above holds.
\item $\beta_{\fa{A}}(S_{\fa{C}}) = S_{\fa{A}}$ and $\beta_{\fa{A}}(F_{\fa{C}})
= F_{\fa{A}}$ holds by definition
\item $\beta_{\fa{A}}\Big(\pathcat{G_{\fa{C}}}(S_{\fa{C}}, F_{\fa{C}})\Big)
\stackrel{!}{=} \pathcat{G_{\fa{A}}}(S_{\fa{A}}, F_{\fa{A}})$
\end{enumerate}

Let $\pi \in \pathcat{G_{\fa{A}}}(S_{\fa{A}}, F_{\fa{A}})$ be an accepting path.
We have to show that there exists an accepting path $\pi' \in
\pathcat{G_{\fa{C}}}(S_{\fa{C}}, F_{\fa{C}})$ with $\beta_{\fa{A}}(\pi') = \pi$.

Because $\gamma_{\fa{A}}$ is a reduction, there exists an accepting path
$\pi_{\fa{D}} \in \pathcat_{\fa{D}}(S_{\fa{D}}, F_{\fa{D}})$ with
$\gamma_{\fa{A}}(\pi) = \pi_{\fa{D}}$.

Further there exists an accepting path $\pi_{\fa{B}} \in
\pathcat{G_{\fa{B}}}(S_{\fa{B}}, F_{\fa{B}})$ with
$\gamma_{\fa{B}}(\pi_{\fa{B}}) = \pi_{\fa{D}}$, because $\gamma_{\fa{B}}$ is a
reduction (and therefore is surjective).

From the paths $\pi$ and $\pi_{\fa{B}}$ we construct the path we are searching
for:
\[ \pi = e_1 \cdots e_n \qquad \pi_{\fa{D}} = \gamma_{\fa{A}}(e_1) \cdots
\gamma_{\fa{A}}(e_n) = \gamma_{\fa{B}}(\pi_{\fa{B}}) \]

Because $\gamma_{\fa{A}}$ and $\gamma_{\fa{B}}$ are length-preserving it follows
\[ \pi_{\fa{B}} = e'_1 \cdots e'_n \]
with $e'_i \in E_{\fa{B}}$.

Now create the path $\pi_{\fa{C}} = (e_1, e'_1) \cdots (e_n, e'_n) \in
\pathcat{G_{\fa{C}}}(S_{\fa{C}}, F_{\fa{C}})$.

\[ \Rightarrow \pathcat{G_{\fa{A}}}(S_{\fa{A}}, F_{\fa{A}}) \subset
\beta_{\fa{A}}(\pathcat_{\fa{C}}(S_{\fa{C}}, F_{\fa{C}})) \]
\[ \Rightarrow \pathcat{G_{\fa{A}}}(S_{\fa{A}}, F_{\fa{A}}) =
\beta_{\fa{A}}(\pathcat_{\fa{C}}(S_{\fa{C}}, F_{\fa{C}})) \]
\end{proof}






















