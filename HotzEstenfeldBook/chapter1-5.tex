\section{Subcategory, generating system}

\begin{definition}
Let $U = (Obj(U), Mor(U), Q_U, Z_U, \circ_U)$ and $C = (Obj(C), Mor(C), Q_C,
Z_C, \circ_C)$ be categories.

$U$ is called a {\bf subcategory} of $C \Leftrightarrow$
\begin{enumerate}
  \item $Obj(U) \subset Obj(C)$ and $Mor(U) \subset Mor(C)$
  \item $Q_U = Q_C|_{Mor(U)}$ and $Z_U = Z_C|_{Mor(U)}$
  \item $\circ_U = \circ_C|_{Mor(U) \times Mor(U)}$
  \item For $w \in Obj(U) \Rightarrow 1_w \in Mor(U)$
\end{enumerate}
\end{definition}

$U$ is called {\bf full subcategory} of $C \Leftrightarrow$
\[ \forall w_1, w_2 \in Obj(U), f: w_1 \to w_2 \in Mor(C) \Rightarrow f \in
Mor(U) \]

This means, all morphisms in $C$ between objects in $U$ are also morphisms in
$U$. $f: w_1 \to w_2$ stands for $Q(f) = w_1 \wedge Z(f) = w_2$.

We want to explain this fact at some examples:

{\bf Example 1}:

Let $A = \{ x,y,z,a,b,c \}$ and $f, g, h: A^* \to A^*$ be mappings defined as
follows:
\[ f(u_1 \cdot x \cdot u_2 \cdot x \cdots x \cdot u_k \cdot x) = 
u_1 \cdot ax \cdot u_2 \cdot ax \cdots ax \cdot u_k \cdot ax) \]

where $u_i \in (A - \{x\})^*$,

\[ g(u_1 \cdot y \cdot u_2 \cdot y \cdots y \cdot u_k \cdot y) = 
u_1 \cdot by \cdot u_2 \cdot by \cdots by \cdot u_k \cdot by) \]

where $u_i \in (A - y\})^*$,

\[ h(u_1 \cdot z \cdot u_2 \cdot z \cdots z \cdot u_k \cdot z) = 
u_1 \cdot cz \cdot u_2 \cdot cz \cdots cz \cdot u_k \cdot cz) \]

where $u_i \in (A - y\})^*$.

Let $M$ be the monoid of mappings generated by $f, g, h : A^* \to A^*$.

Then $C = ({A^*}, M, Q, Z, \circ)$ is a category, if $\circ$ denotes the monoid
operation in $M$.

Let $G$ be the graph defined by the following figure:

FIGURE

We define a functor $\phi = (\mathcal{W}(G), C, \phi_1, \phi_2)$ by $\phi_1(1)
:= A^*$ and $\phi_2(e) := f \circ g \circ h$.

$\phi_2(\mathcal{W}(G))$ is a subcategory of $C$ and it holds:
\[ \phi_2(\mathcal{W}(G))(xyz) = \{ (a^n x b^n y c^n z)\ |\ n \in \mathbb{N} \}
\]

{\bf Example 2}:

We expand on example 1. In addition to $f,g,h$ we have three monoid
homomorphisms $f_1,g_1,h_1$ defined by:
\begin{eqnarray*}
f_1(x) = \epsilon, f_1(u) = u \quad\forall u \in A - \{x\} \\
g_1(y) = \epsilon, g_1(u) = u \quad\forall u \in A - y\} \\
h_1(z) = \epsilon, h_1(u) = u \quad\forall u \in A - z\} \\
\end{eqnarray*}

We extend the graph $G$ as follows to a graph $G_1$:

FIGURE

Consider $\mathcal{W}(G_1)(1,2)$. Then $\mathcal{W}(G_1)(1,2) \cup \{1_1,
1_2\}$ is a subcategory of $\mathcal{W}(G_1)$. In addition, $\mathcal{W}(G)$ is
a subcategory of $\mathcal{W}(G_1)$. 

We extend the functor $\phi$ from example 1
onto $\mathcal{W}(G_1)$ by defining:
\[ \phi_2(e_1) := f_1 \circ g_1 \circ h_1 \]

We get:
\[ \phi_2(\mathcal{W}(G_1)(1,2))(xyz) = \{ (a^n b^n c^n\ |\ n \in \mathbb{N} \}.
\]

{\bf Example 3}:

Let $G$ be defined as follows:

FIGURE

The full subcategory of $\mathcal{W}(G)$ generated by $\{ 1', 2', 3' \}$ is the
path category $\mathcal{W}(G')$ with the following graph $G'$:

FIGURE

As an exercise, one can show:































