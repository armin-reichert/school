\section{Homomorphy and equivalence of finite automata}

We consider again the finite automaton from chapter 2.1 and investigate the
question of structural relationship between finite automata.

We tackle the question how finite automata are related which accept the same
language. We will formalize this by the existence of certain functors between
finite automata.

For making statements about decidability wrt.\ accepted languages it is an
advantage to be able to determine a ''simplest'''' finite acceptor. We will
prove the existence of such a ''simplest'' automaton.

First, we consider some simple decidability questions answered by the following
theorems:

\begin{theorem}
Let $\fa{A}$ be a finite automaton. Then the questions $\falang{A} = \emptyset?$
and $\falang{A} = X^*?$ are decidable.
\end{theorem}

Proof:

''$\falang{A} = \emptyset$?'':

Because of $\falang{A} = \emptyset \Leftrightarrow \pathcat{G}(S, F) =
\emptyset$ this is a decidable problem (reachability in a finite graph). The
formal proof is up to the reader.

''$\falang{A} = X^*$?'':

We gave in chapter 2.1 an effective procedure for constructing a deterministic,
finite automaton $\fa{A}'$ from an automaton $\fa{A}$. From this deterministic
automaton, one can construct an automaton $\fa{B}$ accepting the complement of
$\falang{A}$. From $\falang{B} = \emptyset \Leftrightarrow \falang{A} = X^*$,
and 1) follows the claim.

\begin{definition}
$\fa{A}$ is called {\bf weakly-equivalent} to an automaton $\fa{B}$ if the
accepted languages are equal: \[ \falang{A} = \falang{B} \]
\end{definition}

\begin{theorem}[decidability of weak equivalence]
For finite automata $\fa{A}$ and $\fa{B}$ it is decidable if $\falang{A} =
\falang{B}$.
\end{theorem}

Proof: Without loss of generality we may assume that $\fa{A}$ and $\fa{B}$ are
complete and deterministic. 

From $\fa{A}$ and $\fa{B}$ using our constructions we can directly give automata
for 
\[\bar{\falang{A}}, \bar{\falang{B}}, \bar{\falang{A}} \cap
\falang{B}, \falang{A} \cap \bar{\falang{A}}\]
and also for 
\[ \falang{C} := (\bar{\falang{A}} \cap \falang{B}) \cup (\falang{A} \cap
\bar{\falang{A}}) \]

Because $\falang{A} = \falang{B} \Leftrightarrow \falang{C} = \emptyset$ which
is decidable (theorem 1) we get the proof of theorem 2.

We consider now the homomorphy between finite automata. Structure-preserving
mappings between finite automata are called {\bf automata homomorphisms}.

\begin{definition}
Let $\fa{A}, \fa{A'}$ be finite automata. A functor $\beta = (\beta', \beta'')$
is called an {\bf automata homomorphism}, if
\[ \beta' = (\pathcat{G}, \pathcat{G'}, \beta'_1, \beta'_2) \]
is a functor,
\[ \beta'': X^* \to X'^*	\]
is a monoid homomorphism, and the axioms (A1) and (A2) hold:
\begin{itemize}
  \item[(A1)] The following diagram is commutative:
  
  FIGURE
  
  \item[(A2)] $\beta'_1(S) = S'$ and $\beta'_1(F) = F'$. 
\end{itemize}

$\beta$ is called {\bf length-preserving} $\Leftrightarrow \beta'_2(e) \in
E(G')$ for each edge $e \in E(G)$.

$\beta$ is called {\bf non-expanding} $\Leftrightarrow \beta_2'(e) \in E(G')
\cup \{ 1_e \mid e \in E(G') \}$
\end{definition}

In the following we will always identify $\beta'$ with $\beta$.

It holds:

\begin{theorem}
Given an automata homomorphism $\beta: \fa{A} \to \fa{A'}$. If
\[\beta_2(\pathcat{G}(S,F)) = \pathcat{G'}(S',F')\]
then $\beta''(\falang{A}) = \falang{A'}$.
\end{theorem}






























