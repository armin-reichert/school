\section{Special monoids and the free group}

We have just learned about the syntactic monoid as an example for a monoid.
Further information on the theory of syntactic monoids can be found in
\cite{Salomaa} and \cite{Perrot}.

Let's have a look at more special monoids which we will need again later. To do
so, we introduce the notion of {\bf generated congruence relation}.

Let $A$ be an alphabet and $R = \{ u_i = v_i \ |\ i = 1, \ldots, n,\ u_i, v_i \in A^* \}$\ a set of equations.

Then by the following conditions an congruence relation $\bar{R}$ is uniquely
determined:

\begin{enumerate}
  \item $\{ (u_i, v_i)\ |\ u_i = v_i \in R \} \subset \bar{R}$
  \item $\bar{R}$\ is a congruence relation
  \item $\bar{R} \subset R'$\ for all $R'$\ fulfilling conditions 1) and 2).
\end{enumerate}

$\bar{R}$ is called the {\bf congruence relation generated by} $R$ over $A^*$.

The factor monoid $A^*/\bar{R}$ is named also simply $A^*/R$.

It holds: Words $u, v \in A^*$ are congruent wrt. $\bar{R}$ (Notation: $u
\equiv v (\bar{R})$) iff there exists $n \in \mathbb{N}, u_i \in A^*$ with $u_i
= u_{i,1} \cdot u_{i,2} \cdot u_{i,3}$ such that for $i = 1, \ldots, n$ it
holds:

\begin{enumerate}
  \item $u = u_1, v = u_n$
  \item $u_{i,1} = u_{i+1,1}, u_{i,3} = u_{i+1,3}, (u_{i,2} = u_{i+1,2}) \in R$
  for all $i = 1, \ldots, n-1$.
\end{enumerate}

We say: $v$ is constructed from $u$ by applying the equations from $R$.

The congruence classes of $u \in A^*$ in $A^* / R$ are denoted by $[u]_{A^*/R}$
or just $[u]$.

\begin{definition}
Let $X$ be an alphabet. Define $X^{-1} := \{ x^{-1}\ |\ x \in X \}$ as the set
of formal inverses.
\end{definition}

We can think of $x$ and $x^{-1}$ as corresponding pairs of
brackets as we did in the definition of the Dyck languages in the introduction.

We will now consider different partitionings of $(X \cup X^{-1})^*$ wrt. to
different congruence relations and investigate the corresponding factor monoids.

\begin{definition}
\[ X^{[*]} := (X \cup X^{-1})^* / \{ x x^{-1} = 1\ |\ x \in X \} \]
is called the {\bf H-group}. (The name (H = ''half'') shall remember of
semi-group).
\end{definition}

Now we introduce a special absorbing element $0$ by defining:
\begin{definition}
\[ X^{(*)} := (X \cup X^{-1} \cup \{0\})^* / \{ x x^{-1} = 1, x y^{-1} = 0, 0 z
= z 0 = 0\ |\ x,y \in X, z \in X \cup X^{-1} \ \{0\} \} \]
is called the {\bf polycyclic monoid}.
\end{definition}

Using the naming of the previous section we get:
\[ X^{(*)} = synt_{X^*}(D(X)) \]
which means: the polycyclic monoid is the syntactic monoid of the Dyck language.

\begin{definition}
\[ F(X) := (X \cup X^{-1})^* / \{ x x^{-1} = x^{-1} x = 1 \ |\ x \in X \} \]
\end{definition}
is the {\bf free group} over $X$.

Remark: It holds $D(X) = [1]_{X^{(*)}}$ and $D(X) = [1]_{X^{[*]}}$, which means
the Dyck language is the set of words from $(X \cup X^{-1})^*$ which can be
reduced to the empty word.

In the following we will mainly consider the H-group over $X$.

For $w \in (X \cup X^{-1})^*$ we define the reduced word $|w|$ as follows: If
$w$ does not contain a subword of the form $x x^{-1}$ then $|w| = w$. Otherwise,
replace the leftmost occurence of $x x^{-1}$ by the empty word 1.

This process is called {\bf reduction} and the result is denoted by $\rho(w)$.
One can easily prove:

\begin{lemma}
There exists a minmal number $k \in \mathbb{N}$ with $\rho^k(w) = |w|$. The
number $k$ is called the {\bf reduction length} of $w$. It holds: $\rho(|w|) =
|w|$.
\end{lemma}

\begin{lemma}
\[ [w] = [w'] \in X^{[*]} \Leftrightarrow |w| = |w'|. \]
\end{lemma}

Proof: 

''$\Leftarrow$'':

It holds $w \equiv |w| = |w'| \equiv w' \Rightarrow [w] = [w']$.

''$\Rightarrow''$:

Let $[w] = [w']$. We may assume that $w'$ is created from $w$ by application of
an equation $x x^{-1} = 1$. Let $w = w_1 x x^{-1} w_2$ and $w' = w_1 w_2$.

We show: If $k$ is the reduction length of $w_1$ then $\rho^{k+1}(w) =
\rho^k(w')$\ (thus the reduced words are equal).

Proof by induction over $k$:

$k = 0$: $w_1$ is already reduced, so $\rho(w) = w_1 w_2 = w'$.

$k > 0$: It holds $\rho(w) = \rho(w_1 x x^{-1} w_2), \quad \rho(w') = \rho(w_1
w_2)$. The reduction length of $\rho(w)$ by induction proposition is $k-1$ and
$\rho^k \rho(w) = \rho^{k-1}\rho(w') \Rightarrow$ the reduced word of $w$ and
$w'$ is the same so $|w| = |w'|$.

Remark: Using the same argument one can show that the creation of the reduced
word doen not depend on the order of the reductions.

Therefore the reduced word for a representant of an element of $X^[*]$ is
unique, so we can just speak of ''the'' reduced word in the following.

Remark: These results have been used in \cite{HotzMesserschmidt} to obtain a
space-optimal algorithm for the analysis of the Dyck language.

Similar results also hold for the free group $F(X)$, see \cite{CrowellFox}.

