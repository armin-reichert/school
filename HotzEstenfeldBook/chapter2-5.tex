\section{The rational transductor}

In this section we extend the finite automaton with an output which leads to:

\begin{definition}
\[ \fa{T} = (G, X, S, F, \alpha, \beta) \] is called a {\bf finite transductor}
with input alphabet $X$ and output alphabet $Y$ if $(G, X, S, F, \alpha)$ is a
finite automaton, $G = (V, E)$, and $\beta = (\pathcat{G}, Y^*, \beta_1,
\beta_2)$ is a functor.
\end{definition}

$\fa{T}$ is called {\bf length-preserving} if $|\alpha(e)| = |\beta(e)|$ for all
edges $e \in E$.

$\fa{T}$ computes the {\bf finite transduction}
\begin{eqnarray*}
\tau(\fa{T}) & = & \{ (u, v) \in X^* \times Y^* \mid \mbox{there
exists a path } w \in \pathcat{G}(S, F) \\
& &  \mbox{ with } \alpha(w) = u \mbox{ and } \beta(w) = v \}
\end{eqnarray*}

Remark: For each length-preserving transductor $\fa{T}$ there exists a
transductor $\fa{T'}$ with $|\alpha(e)| = |\beta(e)| = 1$ such that
$\tau(\fa{T}) = \tau(\fa{T'})$.

A finite transduction can equivalently be written as 
\[ \tau(\fa{T})(u) = \beta(\alpha^{-1}(u) \cap \pathcat{G}(S, F)),\ u \in X^* \]

We also speak of {\bf rational transduction} instead of finite transduction.

We give an example for a rational transductor:

Let $\fa{T} = (G, \{x\}, \{a, b\}, \{1\}, \{3, 5\}, \alpha, \beta)$ be given by
the graph

FIGURE

The edges are labeled with the input/output symbols.

$\fa{T}$ realizes the transduction $\tau(\fa{T})$ defined by
\[ \tau(\fa{T})(x^n) = \left\{ 
\begin{array}{l@{\quad}l}
a^n & \mbox{if } n \equiv 1 \bmod 2 \\
b^n & \mbox{if } n \equiv 0 \bmod 2
\end{array}
\right. \]

\begin{lemma}[closure under inversion]
Let $\tau(\fa{T})$ be a rational transduction. Then $(\tau(\fa{T}))^{-1}$ is
also a rational transduction.
\end{lemma}

Proof: Let \[ \fa{T} = (G, X, Y, S, F, \alpha, \beta) \] be a finite tranductor.
Then \[ \fa{T'} = (G, Y, X, S, F, \beta, \alpha) \] is a finite transductor and
\[ \tau(\fa{T'}) = (\tau(\fa{T}))^{-1} \]

We now define the image of a language under rational transduction:

\begin{definition}
Let $L \subset X^*$, then
\begin{eqnarray*}
\tau(\fa{T})(L) & := & \{ v \in Y^* \mid \mbox{there exists } u \in L \mbox{
with } (u, v) \in \tau(\fa{T}) \} \\
& := & \bigcup_{u \in L} \tau(\fa{T})(u)
\end{eqnarray*}
\end{definition}

The following theorem holds:

\begin{theorem}[Regular languages are closed under rational transductions]
Let $L \in REG(X^*)$ be a regular language and $\fa{T}$ be a finite transducer,
then the image of $L$ under $\tau$ is a regular language over $Y^*$:
\[ \tau(\fa{T})(L) \in REG(Y^*) \]
\end{theorem}

Proof: By definition, the image of $L$ can be presented as
\[ \tau(\fa{T})(L) = \beta\big( \alpha^{-1}(L) \cap \pathcat{G}(S, F) \big) \]

$\alpha$ may be continued to a monoid homomorphism from $E^*$, the free monoid
over the edge set of graph $G$, into $X^*$.

We already know:
\begin{eqnarray*}
\alpha^{-1}(L) \in REG(E^*)    & \mbox{Theorem 2, chapter 2.3} \\
\pathcat{G}(S, F) \in REG(E^*) & \mbox{Lemma 4, chapter 2.1} \\
\Rightarrow \alpha^{-1}(L) \cap \pathcat{G}(S, F) \in
REG(E^*) & \mbox{Theorem 1, chapter 2.1} \\
\Rightarrow \beta(\alpha^{-1}(L) \cap \pathcat{G}(S, F)) \in
REG(Y^*) & \mbox{Theorem 7, chapter 2.1}
\end{eqnarray*}

The next theorem gives an insight into the power of rational transductions.

\begin{theorem}
For each $L \in REG(Y^*)$ there exists a length-preserving transductor
$\fa{T}_L$ with \[ \tau(\fa{T}_L)(\{x\}^*) = L \]
\end{theorem}

Proof: $L \in REG(Y^*) \Rightarrow$ there exists a finite automaton $\fa{B} =
(G, Y, S, F, \beta),\ G= (V, E)$ with $L = \beta(\pathcat{G}(S, F))$. For $X =
\{x\}$ set $\alpha(e) = x$ for all edges $e \in E$.

Then $\alpha^{-1}(X^*) \cap \pathcat{G}(S, F) = \pathcat{G}(S, F)$, from which
follows \[\beta(\alpha^{-1}(X^*) \cap \pathcat{G}(S, F)) = L\]

\begin{theorem}
If $\tau_1 : X^* \to Y^*$ and $\tau_2 : Y^* \to Z^*$ are finite,
length-preserving transductions, then $\tau_1 \circ \tau_2 : X^* \to Z^*$ is a
finite transduction.
\end{theorem}

Proof: We want to illustrate the proof geometrically as shown in the following
figure:

FIGURE

(Translator's remark: I changed the names in the diagram and the proof to make
it more readable)

We have the following situation:
\[ \tau_i(u) = \beta^{(i)}\big( {\alpha^{(i)}}^{-1}(u) \cap R_i \big) \]
with $\alpha^{(1)} : E_1^* \to X^*,\quad \beta^{(1)} : E_1^* \to Y^*,\quad
\alpha^{(2)} : E_2^* \to Y^*,\quad \beta^{(2)} : E_2^* \to Z^*$.

Let
\[ E_3 := \{ (e_1, e_2) \mid \beta^{(1)}(e_1) = \alpha^{(2)}(e_2) \} \]
and 
\[ R_3 := \chi^{-1}(R_1) \times \rho^{-1}(R_2) = \pathcat{}(S_1 \times S_2, F_1
\times F_2) \]
























