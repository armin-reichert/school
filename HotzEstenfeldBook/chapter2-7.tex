\section{Regular sets in \pocymon{X}, REG(\pocymon{X})}

In the first section of this chapter we looked at $REG(X^*)$, the class of
regular languages over the free monoid $X^*$. 

We replace now the free monoid by an important special case of a monoid, namely
the polycyclic monoid \pocymon{X} which is the syntactic monoid of the
Dyck-language of the alphabet $X \cup X^{-1}$.

We represent the elements of \pocymon{X} by their reduced words (see chapter
1.3).

\begin{definition}
Let $|w|$ be the reduced word of $w \in (X \cup X^{-1})^*$ with respect to the
polycyclic monoid \pocymon{X} and let $L \subset (X \cup X^{-1})^*$ be a
language. Then
\[ |L| := \{ |w| \mid w \in L,\ |w| \neq 0 \} \]
is the set of reduced words of $L$.
\end{definition}

\begin{definition}
\begin{eqnarray*}
|REG|(\pocymon{X}) & := & \{ L \subset (X\cup X^{-1})^* \mid \mbox{there exists
}L' \in REG(\pocymon{X}) \\
& & \mbox{ with } L = \{|w| \mid [w] \in L' \} \}
\end{eqnarray*}
\end{definition}

Our goal is to show the inclusion
\[ |REG|(\pocymon{X}) \subset REG((X \cup X^{-1})^*) \]

This inclusion does not tell that the word problem of \pocymon{X} is rational,
or in other words that $[w]$ is a rational set. But the structure of
$REG(\pocymon{X})$ is determined for the biggest part by $REG((X \cup
X^{-1})^*)$.

Of course it is
\[ |REG|(\pocymon{X}) \neq REG((X \cup X^{-1})^*) \]

But we can completely characterize $|REG|(\pocymon{X})$ as a subset of $REG((X
\cup X^{-1})^*)$.

To do so, we prove at first a number of lemmata.

\begin{lemma}
\[ L \subset REG((X \cup X^{-1})^*) \Rightarrow |L| \in REG((X \cup X^{-1})^*)
\]
\end{lemma}

\begin{proof}
Let \[ \fa{A} = (G, X \cup X^{-1}, S, F, \alpha) \] be a finite automaton
accepting $L$.

We construct in several steps an automaton $\fa{A}'$ accepting $|L|$.

This construction will play an exceptional role in later chapters.

Without loss of generality we may assume that for each edge of our graph $G =
(V, E)$ holds: $\alpha(e) \in X\cup X^{-1}\cup \{\epsilon\}$.

{\bf Step 1:} If there is a path $\pi \in \pathcat{G}(v, v')$ with
$|\alpha(\pi)| = \epsilon$, then we add a new edge $e: v \to v'$ to $E$ and set
$\alpha(e) := \epsilon$. $\epsilon$-loops will be removed.

We do this for each such pair $(v, v')\in V x V$ and get a graph $G_1$ with
corresponding labelling $\alpha$.

{\bf Step 2:} For edges $e\in E$ with $\alpha(e) = \epsilon$ and $e'\in E$ with
$\alpha(e')\neq \epsilon$ and $Z(e) = Q(e')$ we add an edge $e''$ with
$Q(e'') := Q(e)$, $Z(e'') := Z(e')$ and $\alpha(e'') := \alpha(e')$.

If $Z(e') = Q(\tilde{e})$ with $Z(\tilde{e}) \in F$ is a final state and
$\alpha(\tilde{e}) = \epsilon$, then add edge $e''$ with $Q(e'') := Q(e)$,
$Z(e'') := Z(\tilde{e})$ and $\alpha(e'') := \alpha(e')$.

\transrem{Insert figure to clarify this}

If we do this for all such pairs of edges $(e, e')$, we get a graph $G_2$ with
labelling $\alpha$ as described.

{\bf Step 3:} We remove all edges $e \in G_2$ where $\alpha(e) = \epsilon$. The
resulting graph shall be $G_3$ with labelling $\alpha$.

Because of our construction we get:
\[ \lang{A}_3 = \{ v \mid v \mbox{ is reduced word of } w \in
\lang{A} \mbox{ wrt. } x\cdot x^{-1} = 1 \} \]

{\bf Step 4:} We interrupt all paths with a label containing $x \cdot y^{-1},\
x\neq y$. To do so, we duplicate certain vertices and edges of $G_3$.

Let $v$ be a vertex which is the target of an edge $e$ with $\alpha(e) \in X$
and is the source of an edge $e'$ with $\alpha(e') \in X^{-1}$.

We replace vertex $v$ by vertices $v$ and $v^+$ and add edges as follows:

All edges $e_1$ with $Z(e_1) = v$ and $\alpha(e_1) \in X$ will be rerouted to
vertex $v^+$. Edges $e_2$ with $Z(e_2) = v$ and $\alpha(e_2) \in X^{-1}$ are
left unchanged.

All edges $e_3$ with $Q(e_3) = v$ and $\alpha(e_3) \in X$ will be duplicated
into $e_3^+$ and $e_3^-$. One of them leads from $v^+$ and the other from $v$ to
$Z(e_3)$.

The labelling is defined by $\alpha(e_3^+) = \alpha(e_3^-) = \alpha(e_3)$.

Edges $e_4$ with label $\alpha(e_4) \in X^{-1}$ and $Q(e_4) = v$ stay unchanged.

The following figure shows the construction:

FIGURE

We do this with every such vertex of $G_3$. Tge resulting graph shall be $G'$.
With start and final state sets $S' := S$ and $F' := F \cup \{e^+ \mid e\in F\}$
we get a finite automaton
\[ \fa{A}' = (G', X \cup X^{-1}, ', F', \alpha') \]

It then holds: $\lang{A}' = |L|$!

We observe that $\lang{A}'$ contains only reduced words which are different from
zero. Additionally, the reversal of the construction of $G'$ from $G_3$ is a
functor which preserves the labelling of the edges.

Therefore $\lang{A}' \subset |L|$.

Let to the opposite $\pi$ be a path in $\pathcat{G_3}$ with label $\alpha(\pi)
= |\alpha(\pi)|$, then there is no $(+, -)$-sequence in the label.

That means the path $\pi$ stays unchanged in $G'$ as long as the label of the
edges is in $X^{-1}$.

If the label changes, the path switches to the $+$-vertices and stays
''positive'' until its target. If we denote this path that belongs to $\pi$ by
$w'$, then $\alpha(\pi) = \alpha'(\pi')$ from which follows $|L| \subset
\lang{A}'$.

Together we get $|L| = \lang{A'}$.
\end{proof}

In this proof we even showed more which is summarized in the following
corollary.

\begin{corollary}
To each path $\pi = e_1 \cdots e_k \in \pathcat{G}(S, -),\ e_i \in E$ with
$\alpha(e_k) \neq \epsilon$ and $\alpha(\pi) \neq \epsilon$ there is assigned a
unique $\epsilon$-free path $\pi' \in \pathcat{G_3}(S, -)$ with $\alpha(\pi) =
\alpha(\pi')$.

Additionally, we can define a functor $\nu : \pathcat{G_3} \to \pathcat{G}$ such
that the following holds:

For each accepted word $u \in L$ there exists an accepting path $\pi = e_1
\cdots e_n \in \pathcat{G_3}(S, F)$ with $e_i \in E_3,\ \alpha(e_i) \neq \epsilon$ such
that $\alpha(\nu(\pi)) = u$
\end{corollary}

Proof: Exercise.

Our construction show even more: the number of paths $\pi$ with $\alpha(\pi) =
u$ for a given $u \in |L|$ is the same in $\pathcat{G_3}$ as in $\pathcat{G'}$.
Therefore this number is in $\pathcat{G'}$ less or equal to the number in
$\pathcat{G_2}$. A similar statement holds for $\pathcat{G_1}$ and
$\pathcat{G}$.

This fact, which will be of importance later in this book, is summarized in the
following corollary. We define

\begin{definition}
\[ INDEX(|\fa{A}|, u) := card( \{ \pi \in \pathcat{G}(S, F) \mid |\alpha(\pi) =
u \} ) \]
\end{definition}

\begin{corollary}
\[ INDEX(|\fa{A}|, u) \geq INDEX(|\fa{A}'|, u),\ u \in \lang{A}' \]
\end{corollary}

Remark: Because $\lang{A}' = |\lang{A}'|$ it holds $INDEX(|\fa{A}'|, u) =
INDEX(\fa{A}', u)$.

Now we can prove the announced theorem.

\begin{theorem}
\[ |REG|(\pocymon{X}) = \{ |L| \mid L \in REG((X\cup X^{-1})^*) \} \]
\end{theorem}

\begin{proof}
 The inclusion ''$\supset$'' is clear.
 
 ''$\subset$'': Every functor $\alpha : \pathcat{G} \to \pocymon{X}$ can be
 extended to a functor $\alpha' : \pathcat{G} \to (X \cup X^{-1})^*$ by
 defining $\alpha'(e) := |u|$ for each edge $e \in E$ and $\alpha(e) = [u]$.
 
 $\alpha'$ is continued to a functor and for the finite automaton
 \[ \fa{A}' = (G', X \cup X^{-1}, S', F', \alpha') \]
 it holds:
 \[ \lang{A} = \{ [u] \mid u \in \lang{A}' \} \]
 and with lemma 1 we get:
 \[ \lang{A} = \{ [u] \mid u \in |\lang{A}'| \} \]
 which means
 \[ |\lang{A}| = \{ |u| \mid u \in \lang{A}' \} \]
\end{proof}

The next theorem gives an even more precise characterization for
$|REG|(\pocymon{X})$.

\begin{theorem}
Let $\fa{A}$ be a finite automaton
\[ \fa{A} = (G, X \cup X^{-1}, S, F, \alpha) \]
where $G = (V, E)$ and $card(E) = c$. Then there exist $c$ pairs $(L_i^+,
L_i^-)$ with 
\[ L_i^+ \in REG(X^*),\qquad L_i^- \in REG((X^{-1})^*) \]
such that
\[ |\lang{A}| = \bigcup_{i=1}^c L_i^- \cdot L_i^+ \]
\end{theorem}

\begin{proof}
The proof follows directly from the construction of the automaton $\fa{A}'$.

If $V'$ is the set of the vertices $v$ which have an associated vertex $v^+$ in
$G'$, then every path $\pi \in \pathcat{G'}(S', F')$ can be uniquely split into
a product $\pi_1 \cdot \pi_2$ where the target $Z(\pi_1)$ is such a vertex $v$.

Therefore one can write
\[ |\lang{A}| = \bigcup_{v \in V} L_{\fa{A}(S, v)}^- \cdot L_{\fa{A}(v, F)}^+
\]
\end{proof}

\begin{corollary}
\[ INDEX(\fa{A}', u) = \sum_{v \in V' \atop u_1 \cdot u_2 = u}
INDEX(\fa{A}'(S, v), u_1) \cdot INDEX(\fa{A}'(v, F), u_2) \]
\end{corollary}

\begin{lemma}
Let $L = \lang{A}^+ \subset X^*$ and $L' = \lang{A'}^- \subset (X^{-1})^*$.
\end{lemma}


















