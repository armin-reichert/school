\section{The rational transducer}

In this section we extend the finite automaton with an output which leads to:

\begin{definition}
\[ \fa{T} = (G, X, S, F, \alpha, \beta) \] is called a {\bf finite transducer}
with input alphabet $X$ and output alphabet $Y$ if $(G, X, S, F, \alpha)$ is a
finite automaton, $G = (V, E)$, and $\beta = (\pathcat{G}, Y^*, \beta_1,
\beta_2)$ is a functor.
\end{definition}

$\fa{T}$ is called {\bf length-preserving} if $|\alpha(e)| = |\beta(e)|$ for all
edges $e \in E$.

$\fa{T}$ computes the {\bf finite transduction}
\begin{eqnarray*}
\tau(\fa{T}) & = & \{ (u, v) \in X^* \times Y^* \mid \mbox{there
exists a path } w \in \pathcat{G}(S, F) \\
& &  \mbox{ with } \alpha(w) = u \mbox{ and } \beta(w) = v \}
\end{eqnarray*}

Remark: For each length-preserving transducer $\fa{T}$ there exists a
transducer $\fa{T'}$ with $|\alpha(e)| = |\beta(e)| = 1$ such that
$\tau(\fa{T}) = \tau(\fa{T'})$.

A finite transduction can equivalently be written as 
\[ \tau(\fa{T})(u) = \beta(\alpha^{-1}(u) \cap \pathcat{G}(S, F)),\ u \in X^* \]

We also speak of {\bf rational transduction} instead of finite transduction.

We give an example for a rational transducer:

Let $\fa{T} = (G, \{x\}, \{a, b\}, \{1\}, \{3, 5\}, \alpha, \beta)$ be given by
the graph

FIGURE

The edges are labeled with the input/output symbols.

$\fa{T}$ realizes the transduction $\tau(\fa{T})$ defined by
\[ \tau(\fa{T})(x^n) = \left\{ 
\begin{array}{l@{\quad}l}
a^n & \mbox{if } n \equiv 1 \bmod 2 \\
b^n & \mbox{if } n \equiv 0 \bmod 2
\end{array}
\right. \]

\begin{lemma}[closure under inversion]
Let $\tau(\fa{T})$ be a rational transduction. Then $(\tau(\fa{T}))^{-1}$ is
also a rational transduction.
\end{lemma}

Proof: Let \[ \fa{T} = (G, X, Y, S, F, \alpha, \beta) \] be a finite tranductor.
Then \[ \fa{T'} = (G, Y, X, S, F, \beta, \alpha) \] is a finite transducer and
\[ \tau(\fa{T'}) = (\tau(\fa{T}))^{-1} \]

We now define the image of a language under rational transduction:

\begin{definition}
Let $L \subset X^*$, then
\begin{eqnarray*}
\tau(\fa{T})(L) & := & \{ v \in Y^* \mid \mbox{there exists } u \in L \mbox{
with } (u, v) \in \tau(\fa{T}) \} \\
& := & \bigcup_{u \in L} \tau(\fa{T})(u)
\end{eqnarray*}
\end{definition}

The following theorem holds:

\begin{theorem}[Regular languages are closed under rational transductions]
Let $L \in REG(X^*)$ be a regular language and $\fa{T}$ be a finite transducer,
then the image of $L$ under $\tau$ is a regular language over $Y^*$:
\[ \tau(\fa{T})(L) \in REG(Y^*) \]
\end{theorem}

Proof: By definition, the image of $L$ can be presented as
\[ \tau(\fa{T})(L) = \beta\big( \alpha^{-1}(L) \cap \pathcat{G}(S, F) \big) \]

$\alpha$ may be continued to a monoid homomorphism from $E^*$, the free monoid
over the edge set of graph $G$, into $X^*$.

We already know:
\begin{eqnarray*}
\alpha^{-1}(L) \in REG(E^*)    & \mbox{Theorem 2, chapter 2.3} \\
\pathcat{G}(S, F) \in REG(E^*) & \mbox{Lemma 4, chapter 2.1} \\
\Rightarrow \alpha^{-1}(L) \cap \pathcat{G}(S, F) \in
REG(E^*) & \mbox{Theorem 1, chapter 2.1} \\
\Rightarrow \beta(\alpha^{-1}(L) \cap \pathcat{G}(S, F)) \in
REG(Y^*) & \mbox{Theorem 7, chapter 2.1}
\end{eqnarray*}

The next theorem gives an insight into the power of rational transductions.

\begin{theorem}
For each $L \in REG(Y^*)$ there exists a length-preserving transducer
$\fa{T}_L$ with \[ \tau(\fa{T}_L)(\{x\}^*) = L \]
\end{theorem}

Proof: $L \in REG(Y^*) \Rightarrow$ there exists a finite automaton $\fa{B} =
(G, Y, S, F, \beta),\ G= (V, E)$ with $L = \beta(\pathcat{G}(S, F))$. For $X =
\{x\}$ set $\alpha(e) = x$ for all edges $e \in E$.

Then $\alpha^{-1}(X^*) \cap \pathcat{G}(S, F) = \pathcat{G}(S, F)$, from which
follows \[\beta(\alpha^{-1}(X^*) \cap \pathcat{G}(S, F)) = L\]

\begin{theorem}
If $\tau_1 : X^* \to Y^*$ and $\tau_2 : Y^* \to Z^*$ are finite,
length-preserving transductions, then $\tau_1 \circ \tau_2 : X^* \to Z^*$ is a
finite transduction.
\end{theorem}

Proof: We want to illustrate the proof geometrically as shown in the following
figure:

\begin{center}
\begin{tikzcd}[row sep=large]
& & E_3^* \cap R_3 \arrow[dl, "\chi"'] \arrow[dr, "\rho"] & & \\
& E_1^* \cap R_1 \arrow[dl, "\alpha^{(1)}"'] \arrow[dr, "\beta^{(1)}"] & & E_2^*
\cap R_2 \arrow[dl, "\alpha^{(2)}"'] \arrow[dr, "\beta^{(2)}"] & \\
X^* \arrow[rr, "\tau_1"] & & Y^* \arrow[rr, "\tau_2"] & & Z^*
\end{tikzcd}
\end{center}

(Translator's remark: I slightly changed names to make the proof more readable)

We have the following situation:
\[ \tau_1(u) = \beta^{(1)}\big( {\alpha^{(1)}}^{-1}(u) \cap R_1 \big) \]
and
\[ \tau_2(v) = \beta^{(2)}\big( {\alpha^{(2)}}^{-1}(v) \cap R_2 \big) \]

with $u \in X^*, v \in Y^*$ and $R_1 \subset E_1^*, R_2 \subset E_2^*$ regular
languages and monoid homomorphisms
\[ \alpha^{(1)} : E_1^* \to X^*,\quad \beta^{(1)} : E_1^* \to
Y^*,\quad \alpha^{(2)} : E_2^* \to Y^*,\quad \beta^{(2)} : E_2^* \to Z^* \]

Let
\[ E_3 := \{ (e_1, e_2) \mid \beta^{(1)}(e_1) = \alpha^{(2)}(e_2) \} \]
and 
\[ R_3 := \chi^{-1}(R_1) \times \rho^{-1}(R_2) = \pathcat{}(S_1 \times S_2, F_1
\times F_2) \]

with monoid homomorphisms $\chi: e_3^* \to E_1^*$ and $\rho: E_3^* \to E_2^*$.

Now let $(u, v) \in \tau_1$ and ($v, w) \in \tau_2$. Then there exists paths
$\omega_1 \in E_1^* \cap R_1$ and $\omega_2 \in E_2^* \cap R_2$ such that 
\[ u = \alpha^{(1)}(\omega_1),\quad v = \beta^{(1)}(\omega_1) =
\alpha^{(2)}(\omega_2),\quad w = \beta^{(2)}(\omega_2) \]

The ''parallel'' path $\omega = (\omega_1(1), \omega_2(1)) \cdot
(\omega_1(n), \omega_2(n))$ is an element of $E_3^* \cap R_3$ and 
$\chi(\omega) = \omega_1,\ \rho(\omega) = \omega_2$ (without loss of
generality the length-restriction from the remark after definition 1 should
hold here too).

Together this means
\[ (u, v) \in \rho \circ \beta^{(2)} \big( {\alpha^{(1)}}^{-1} \circ
\chi^{-1}(X^*) \cap R_3 \big) \]
which means $(u, v) \in \tau_3$.

To the opposite, let $(u, v) \in \tau_3$. Then there exists a path $\omega \in
E_3^* \cap R_3$ with $\chi(\alpha^{(1)})(\omega) = u$ and $\rho(\beta^{(2)})(\omega)
= v$.

Let $\omega_1 = \chi(\omega)$ and $\omega_2 = \rho(\omega)$ (''projections''), then we get
\[ \omega_1 \in E_1^* \cap R_1,\qquad \omega_2 \in E_2^* \cap R_2 \]

By construction of $E_3$ it follows
\[ \beta^{(1)}(\omega_1) = \alpha^{(2)}(\omega_2) \]

\begin{corollary}
Length-preserving transductions with composition form a category.
\end{corollary}

We generalize now our statement by allowing $\alpha$ and $\beta$ to be arbitrary
functors to $X^*$ resp. $Y^*$.

Next, we investigate the case $|\alpha(w)| \leq |w|$ and $|\beta(w)| \leq |w|$.
Afterwards we will reduce the general case to this special case.

Notation: Functors fulfilling the condition above are called {\bf
non-expanding}.

\begin{theorem}[Composition of non-expanding transductions is a transduction]
Let $\fa{T}_i = (G_i, X_i, S_i, F_i, \alpha^{(i)}, \beta^{(i)}),\ G_i = (V_i,
E_i)$ be transducers with non-expanding functors $\alpha^{(i)}$ and
$\beta^{(i)},\ i = 1,2.$

If $Y_1 = X_2$ (the input alphabet of the second is the output alphabet of the
first transducer), then the composition $\tau(\fa{T}_1) \circ \tau(\fa{T}_2)$
is a transduction.
\end{theorem}

Proof: Our assumption is described by the following figure:

\begin{center}
\begin{tikzcd}[row sep=large]
& E_1^* \cap \pathcat{G_1}(S_1,F_1) \arrow[dl,"\alpha^{(1)}"']
\arrow[dr,"\beta^{(1)}"] & & E_2^* \cap \pathcat{G_2}(S_2, F_2)
\arrow[dl,"\alpha^{(2)}"'] \arrow[dr,"\beta^{(2)}"] & \\
X_1^* \arrow[rr,"\tau(\fa{T}_1)"] & & Y_1^* \arrow[rr,"\tau(\fa{T}_2)"] & &
Y_2^*
\end{tikzcd}
\end{center}

Remark: It should be clear that the addition of $\epsilon$-loops ($e \in E,
\alpha(e) = \beta(e) = \epsilon$) in the graph does not change the computed
transduction.

We add to each state of $\fa{T}_1$ and $\fa{T}_2$ such a loop.

Because this does not change the computed transductions, this also is the case
for the composition of those.

We define the vertex set of the composed transducer as $V := V_1 \times V_2$.
The edge set is given by
\[ E := \{ (e_1, e_2) \in E_1 \times E_2 \mid \alpha^{(2)}(e_2) =
\beta^{(1)}(e_1) \} \]
where
\begin{eqnarray*}
Q((e_1, e_2)) & = & (Q(e_1), Q(e_2)) \\
Z((e_1, e_2)) & = & (Z(e_1), Z(e_2)) 
\end{eqnarray*}

With the projections $\pi^i : E^* \to E_i^*,\ i = 1,2$ we get the commutative
diagram:

\begin{center}
\begin{tikzcd}
& E^* \arrow[dl, "\pi^1"'] \arrow[dd, "\sigma"] \arrow[dr,"\pi^2"] & \\
E_1^* \arrow[dr, "\beta^{(1)}"'] & & E_2^* \arrow[dl, "\alpha^{(2)}"] \\
& Y_1^* &
\end{tikzcd}
\end{center}

and it holds: for $v \in \beta^{(1)}(E_1^*) \cap \alpha^{(2)}(E_2^*)$ there
exists a path $u \in E^*$ with $\sigma(u) = v$.

Define start and final state sets as follows:
\begin{eqnarray*}
S & = & S_1 \times S_2 \\
F & = & F_1 \times F_2 \\
\end{eqnarray*}
Then the composed transducer
\[\fa{T} = (G, X_1, Y_2, S, F, \pi^1 \circ \alpha^{(1)}, \pi^2 \circ
\beta^{(2)})\]
computes the composition of transductions $\tau_1$ and $\tau_2$:
\[ \tau(\fa{T}) = \tau(\fa{T}_1) \circ \tau(\fa{T}_2) \]

(Exercise).

In analogy to the corresponding result for finite automata it holds:

\begin{lemma}
For each transducer
\[\fa{T} = (G, X, Y, S, F, \alpha, \beta)\] there exists a
transducer \[\fa{T'} = (G', X, Y, S', F', \alpha', \beta')\] with
\begin{eqnarray*}
& \forall e \in E(G'): \alpha'(e) \in X \cup \{\epsilon\} \\
& \forall e \in E(G'): \beta'(e) \in Y \cup \{\epsilon\} \\
& card(S') = 1 \\
& card(F') = 1
\end{eqnarray*}
which computes the same transduction as $\fa{T}$.
\end{lemma}

Proof: Let $\fa{T}$ be an arbitrary transducer. If an edge $e \in E$ has an
input or output label longer than 1, that is $e \in E$ with $|\alpha(e)| = k >
1$ or $|\beta(e)| = l > 1$ and $m := max(k, l)$, then a path of length $m$ is
split as follows:
\[ \edge{e_1} \edge{e_2} \ldots \edge{e_m} \]

\begin{enumerate}
  \item $|\alpha'(e_i)| \leq 1$
  \item $|\beta'(e_i)| \leq 1,\quad i = 1,\ldots,m$
  \item $\alpha(e) = \alpha'(e_1 \cdots e_m)$ and $\beta(s) = \beta'(e_1
  \cdots e_m)$
\end{enumerate}

For the resulting transducer $\fa{T'}$ it holds \[ \tau(\fa{T}) = \tau(\fa{T}')
\]

The construction for making this transducer initial (single start state) is
completely analog to the construction in lemma 1 in chapter 2.1.

From lemma 2 and theorem 4 we get directly

\begin{theorem}[composition of rational transductions is a rational
transduction] For rational transductions $\tau_1$ and $\tau_2$ the composition $\tau_1 \circ
\tau_2$, if defined, is also a rational transduction.
\end{theorem}

Remark: An equivalent definition of rational transductions can be given using
matrices. These are a compact description for transductions which one gets by
putting the output words that belong to a fixed input word into the matrix, for
all state changes.

Multiplication of matrices corresponds to concatenation of paths in the graph of
the transducer and to the union of sets of output words of these paths.

The interested reader should refer to \cite{Be} for more information.

















